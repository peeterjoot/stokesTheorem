%
% Copyright © 2016 Peeter Joot.  All Rights Reserved.
% Licenced as described in the file LICENSE under the root directory of this GIT repository.
%
\makeproblem{\R{3} dual forms of Stokes' theorem.}{problem:stokesTheoremCoreProblems:1}{
Prove
\makesubproblem{}{problem:stokesTheoremCoreProblems:1:a}
\cref{eqn:stokesTheoremCore:1681},
\makesubproblem{}{problem:stokesTheoremCoreProblems:1:b}
\cref{eqn:stokesTheoremCore:1801},
\makesubproblem{}{problem:stokesTheoremCoreProblems:1:c}
and \cref{eqn:stokesTheoremCore:1801c}.
} % problem
\makeanswer{problem:stokesTheoremCoreProblems:1}{
The volume elements are
\makeSubAnswer{}{problem:stokesTheoremCoreProblems:1:a}
\begin{subequations}
\label{eqn:stokesTheoremCoreProblems:20}
\begin{equation}\label{eqn:stokesTheoremCoreProblems:40}
\begin{aligned}
d^2 \Bx \cdot \spacegrad
&= dA \gpgradeone{ I \ncap \spacegrad } \\
&= dA I \ncap \wedge \spacegrad
\end{aligned}
\end{equation}
\begin{equation}\label{eqn:stokesTheoremCoreProblems:60}
\begin{aligned}
d^2 \Bx \cdot (\spacegrad \wedge \BA)
&= dA \gpgradezero{ I \ncap \spacegrad \BA } \\
&= dA I \ncap \wedge \spacegrad \wedge \BA
\end{aligned}
\end{equation}
\begin{equation}\label{eqn:stokesTheoremCoreProblems:80}
\begin{aligned}
d^3 \Bx \cdot \spacegrad \phi
&= dV \gpgradetwo{ I \spacegrad \phi } \\
&= dV I \spacegrad \phi
\end{aligned}
\end{equation}
\begin{equation}\label{eqn:stokesTheoremCoreProblems:100}
\begin{aligned}
d^3 \Bx \cdot (\spacegrad \wedge \BA)
&= dV \gpgradeone{ I (\spacegrad \wedge \BA) } \\
&= dV I \spacegrad \wedge \BA
\end{aligned}
\end{equation}
\begin{equation}\label{eqn:stokesTheoremCoreProblems:120}
\begin{aligned}
d^3 \Bx \cdot (\spacegrad \wedge B)
&= dV \gpgradezero{ I (\spacegrad \wedge B) } \\
&= dV I \spacegrad \wedge B.
\end{aligned}
\end{equation}
\end{subequations}

The corresponding boundary forms are
\begin{subequations}
\label{eqn:stokesTheoremCoreProblems:140}
\begin{equation}\label{eqn:stokesTheoremCoreProblems:160}
d\Bx \psi
\end{equation}
\begin{equation}\label{eqn:stokesTheoremCoreProblems:180}
d\Bx \cdot \BA
\end{equation}
\begin{equation}\label{eqn:stokesTheoremCoreProblems:200}
d^2 \Bx \psi
=
dA I \ncap \psi
\end{equation}
\begin{equation}\label{eqn:stokesTheoremCoreProblems:220}
\begin{aligned}
d^2 \Bx \cdot \BA
&= dA \gpgradeone{ I \ncap \BA } \\
&= dA I \ncap \wedge \BA
\end{aligned}
\end{equation}
\begin{equation}\label{eqn:stokesTheoremCoreProblems:240}
\begin{aligned}
d^2 \Bx \cdot B
&= dA \gpgradezero{ I \ncap B } \\
&= dA I \ncap \wedge B.
\end{aligned}
\end{equation}
\end{subequations}

Assembling these pieces back into the integrals proves the relationships.
\makeSubAnswer{}{problem:stokesTheoremCoreProblems:1:b}
To show \cref{eqn:stokesTheoremCore:1841} note that

\begin{equation}\label{eqn:stokesTheoremCoreProblems:260}
\begin{aligned}
I (\Ba \wedge \Bb \wedge \Bc)
&= \gpgradezero{ I \Ba \wedge \Bb \wedge \Bc } \\
&= \gpgradezero{ I \Ba (\Bb \wedge \Bc) - I \Ba \cdot (\Bb \wedge \Bc) } \\
&= \gpgradezero{ I \Ba I(\Bb \cross \Bc) } \\
&= - \Ba \cdot (\Bb \cross \Bc).
\end{aligned}
\end{equation}

To show \cref{eqn:stokesTheoremCore:1901} note that
\begin{equation}\label{eqn:stokesTheoremCoreProblems:280}
\begin{aligned}
\Ba \wedge (I \BA)
&= \Ba \wedge (I \BA) \\
&= \gpgradethree{ \Ba I \BA } \\
&= \gpgradethree{ I \Ba \cdot \BA } \\
&= I (\Ba \cdot \BA).
\end{aligned}
\end{equation}
\makeSubAnswer{}{problem:stokesTheoremCoreProblems:1:c}
For vector \( \Ba \), these transformations all follow from
\begin{equation}\label{eqn:stokesTheoremCoreProblems:300}
\begin{aligned}
\Ba \cross \Bf
&= \gpgradeone{ -I \Ba \wedge \Bf} \\
&= \gpgradeone{ -I \Ba \Bf} \\
&= -\gpgradeone{ \Ba I \Bf} \\
&= -\Ba \cdot (I \Bf) \\
&= \Ba \cdot B.
\end{aligned}
\end{equation}
} % answer
