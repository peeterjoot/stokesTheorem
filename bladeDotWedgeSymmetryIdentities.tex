%
% Copyright � CCYY Peeter Joot.  All Rights Reserved.
% Licenced as described in the file LICENSE under the root directory of this GIT repository.
%

\makeproblem{Blade dot and wedge product reductions}{problem:bladeDotWedgeSymmetryIdentities:1}{
Prove \cref{thm:bladeDotWedgeSymmetryIdentities:180}.
} % problem

\makeanswer{problem:bladeDotWedgeSymmetryIdentities:1}{

The vector product \( B \Ba \) can have only grades \( k-1, k+1 \), so can be written as

\begin{dmath}\label{eqn:bladeDotWedgeSymmetryIdentities:20}
\begin{aligned}
B \Ba &= B \cdot \Ba + B \wedge \Ba \\
\Ba B &= \Ba \cdot B + \Ba \wedge B.
\end{aligned}
\end{dmath}

Suppose that a factorization for the blade \( B \) is found of the form

\begin{dmath}\label{eqn:bladeDotWedgeSymmetryIdentities:40}
B = \Bb_1 \wedge \Bb_2 \wedge \cdots \wedge \Bb_k.
\end{dmath}

Then by successive anticommutation of the wedge products

\begin{dmath}\label{eqn:bladeDotWedgeSymmetryIdentities:60}
\Ba \wedge B = (-1)^k B \wedge \Ba.
\end{dmath}

The dot products from each direction can also be related

\begin{subequations}
\label{eqn:bladeDotWedgeSymmetryIdentities:80}
\begin{equation}\label{eqn:bladeDotWedgeSymmetryIdentities:100}
\begin{aligned}
\Ba \cdot B
&= (\Ba \cdot \Bb_1) (\Bb_2 \wedge \Bb_3 \wedge \cdots \wedge \Bb_k) \\
&\quad- (\Ba \cdot \Bb_2) (\Bb_1 \wedge \Bb_3 \wedge \cdots \wedge \Bb_k) \\
&\quad+ \cdots \\
&\quad+ (-1)^{k-1} (\Ba \cdot \Bb_k) (\Bb_1 \wedge \Bb_2 \wedge \cdots \wedge \Bb_{k-1})
\end{aligned}
\end{equation}
\begin{equation}\label{eqn:bladeDotWedgeSymmetryIdentities:120}
\begin{aligned}
B \cdot \Ba
&= (\Bb_1 \wedge \Bb_2 \wedge \cdots \wedge \Bb_{k-1}) (\Bb_k \cdot \Ba) \\
&\quad-(\Bb_1 \wedge \Bb_2 \wedge \cdots \wedge \Bb_{k-2} \wedge \Bb_k) (\Bb_{k-1} \cdot \Ba) \\
&\quad+ \cdots \\
&\quad+ (-1)^{k-1} (\Bb_2 \wedge \Bb_3 \wedge \cdots \wedge \Bb_{k}) (\Bb_1 \cdot \Ba)
\end{aligned}
\end{equation}
\end{subequations}

Comparision shows that

\begin{dmath}\label{eqn:bladeDotWedgeSymmetryIdentities:140}
\Ba \cdot B = (-1)^{k-1} B \cdot \Ba.
\end{dmath}

The second of the pair of products of \cref{eqn:bladeDotWedgeSymmetryIdentities:20} can now be rewritten

\begin{dmath}\label{eqn:bladeDotWedgeSymmetryIdentities:160}
\begin{aligned}
B \Ba &= B \cdot \Ba + B \wedge \Ba \\
(-1)^k \Ba B &= - B \cdot \Ba + B \wedge \Ba.
\end{aligned}
\end{dmath}

Adding these gives

\begin{dmath}\label{eqn:bladeDotWedgeSymmetryIdentities:240}
\begin{aligned}
B \wedge \Ba &= \inv{2}\lr{ B \Ba + (-1)^k \Ba B } \\
B \cdot \Ba &= \inv{2}\lr{ B \Ba - (-1)^k \Ba B },
\end{aligned}
\end{dmath}

} % answer
