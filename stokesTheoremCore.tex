%
% Copyright © 2016 Peeter Joot.  All Rights Reserved.
% Licenced as described in the file LICENSE under the root directory of this GIT repository.
%
\section{Stokes Theorem}

Stokes theorem is fairly easy to state, but takes a fair amount of work to understand and unpack its implications.

\input{../../physicsplay/notes/gabookI/calculus/stokesTheoremTheStatement.tex}

\section{One parameter specialization of Stokes' theorem.}

An example parameterization with one parameter, and the corresponding differential with respect to that parameter, is plotted in
\cref{fig:oneParameterDifferential:oneParameterDifferentialFig1}.

\imageFigure{../figures/oneParameterDifferentialFig1}{One parameter manifold.}{fig:oneParameterDifferential:oneParameterDifferentialFig1}{0.3}

The differential with respect to the parameter \( a \) is

\begin{equation}\label{eqn:stokesTheoremCore:20}
d\Bx = \PD{a}{\Bx} da \equiv \Bx^a da.
\end{equation}

On this curve the projection of the gradient has just one component

\begin{equation}\label{eqn:stokesTheoremCore:40}
\boldpartial = \Bx^a \partial_a.
\end{equation}

Stokes theorem for a one parameter manifold can only be expressed for scalar fields.  That is

\begin{dmath}\label{eqn:stokesTheoremCore:60}
\int d\Bx \cdot (\boldpartial \wedge \phi)
=
\int d\Bx \cdot \boldpartial \phi
=
\int da \PD{a}{ \phi }
= \evalbar{\phi}{\Delta a}.
\end{dmath}

Observe that the vector derivative can be replaced by the gradient since \( d\Bx \cdot \boldpartial = d\Bx \cdot \spacegrad \).
This is the case since dotting the
gradient with a differential element \( d\Bx \) on this curve, no component of the gradient that isn't colinear to the curve makes no contribution.

That means that Stokes' theorem for a one parameter curve is exactly the fundamental theorem of calculus

\begin{dmath}\label{eqn:stokesTheoremCore:80}
\int_{\Ba}^{\Bb} d\Bx \cdot \spacegrad \phi = \phi(\Bb) - \phi(\Ba).
\end{dmath}

\section{Helper theorem}

\input{../../physicsplay/notes/gabookI/appendix/wedgeDistributionIdentity.tex}
