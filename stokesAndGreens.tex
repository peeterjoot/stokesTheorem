%
% Copyright � CCYY Peeter Joot.  All Rights Reserved.
% Licenced as described in the file LICENSE under the root directory of this GIT repository.
%
\makeproblem{Stokes' theorem relation to Green's theorem}{problem:stokesAndGreens:1}{
Show that Stokes theorem, in its two parameter form, applied to a vector field recovers Green's theorem.
} % problem

\makeanswer{problem:stokesAndGreens:1}{

To demonstrate this, expand the LHS of the Stokes identity

\begin{dmath}\label{eqn:stokesAndGreens:20}
\int_A d^2 \Bx \cdot (\boldpartial \wedge \Bf) = \ointclockwise d\Bx \cdot \Bf.
\end{dmath}

Assuming \( u, v\) parameterization

\begin{dmath}\label{eqn:stokesAndGreens:40}
\int_A d^2 \Bx \cdot (\boldpartial \wedge \Bf)
=
\int_A (d\Bx_u \wedge d\Bx_v) \cdot (\boldpartial \wedge \Bf)
=
\int_A ((d\Bx_u \wedge d\Bx_v) \cdot \Bx^u) \cdot \partial_u \Bf
+
\int_A ((d\Bx_u \wedge d\Bx_v) \cdot \Bx^v) \cdot \partial_v \Bf
=
-\int_A du dv \Bx_v \cdot \partial_u \Bf
+
\int_A du dv \Bx_u \cdot \partial_v \Bf
=
-\int_A du dv \Bx_v \cdot \partial_u \Bf
+
\int_A du dv \lr{
-\Bx_v \cdot \partial_u \Bf
+
\Bx_u \cdot \partial_v \Bf
}.
\end{dmath}

The coordinate expansion of \( \Bf \) with respect to the tangent space coordinates is

\begin{dmath}\label{eqn:stokesAndGreens:60}
\Bf = \Bx^u f_u + \Bx^v f_v + \Bf_\perp
\end{dmath}

where \( \Bf_\perp \) lies in normal to the tangent space at the point in question.  Because \( \Bx_v \) is computed with \( u \) held fixed and \( \Bx_u \) computed with \( v \) held fixed, the area integrand can be written

\begin{dmath}\label{eqn:stokesAndGreens:80}
-\Bx_v \cdot \partial_u \Bf
+
\Bx_u \cdot \partial_v \Bf
=
-\PD{u}{}\lr{ \Bx_v \cdot \Bf }
+\PD{v}{}\lr{ \Bx_u \cdot \Bf }
=
-\PD{u}{f_v}
+\PD{v}{f_u},
\end{dmath}

which gives
\begin{dmath}\label{eqn:stokesAndGreens:100}
\int_A du dv \lr{ -\PD{u}{f_v}
+\PD{v}{f_u}
}
=
\ointclockwise d\Bx \cdot \Bf,
\end{dmath}

which recovers \cref{thm:stokesTheoremGeometricAlgebra:1660} as desired.
} % answer
